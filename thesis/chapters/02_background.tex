\chapter{Background and Related Work}

\section{Evolution of CLI Design}

\subsection{Generation Zero: Manual Parsing}
Early CLIs used raw string parsing, leading to fragile, error-prone interfaces:
\begin{itemize}
    \item Manual argument tokenization
    \item String-based type coercion
    \item Limited validation
    \item Poor ergonomics
\end{itemize}

\subsection{Generation One: Structured Builders}
Libraries like \texttt{clap} introduced builder patterns for type-safe CLI construction:
\begin{lstlisting}[language=Rust]
let app = Command::new("myapp")
    .subcommand(Command::new("verb")
        .arg(Arg::new("noun")))
\end{lstlisting}

\textbf{Limitation}: Entire structure known at compile time.

\subsection{Generation Two: Macro-Based Generation}
Derive macros enabled declarative CLI definitions:
\begin{lstlisting}[language=Rust]
#[derive(Parser)]
struct Args {
    #[command(subcommand)]
    command: Commands,
}
\end{lstlisting}

\textbf{Limitation}: Code generation occurs at compile time; runtime changes impossible.

\subsection{Generation Three: Runtime-First Design (This Work)}
First approach enabling complete CLI structure generation at runtime with zero-cost abstractions.

\section{Model Context Protocol}

MCP establishes a standardized protocol for AI agents to:
\begin{itemize}
    \item Discover available tools/capabilities
    \item Invoke tools with structured arguments
    \item Receive structured results
    \item Coordinate with other agents
\end{itemize}

\subsection{MCP Architecture}
\begin{figure}[H]
\centering
\begin{verbatim}
┌─────────────────────────────────────┐
│      AI Agent (Claude, etc.)        │
└──────────────┬──────────────────────┘
               │ MCP Protocol
┌──────────────▼──────────────────────┐
│  Capability Discovery / Invocation   │
├─────────────────────────────────────┤
│  Tools | Resources | Sampling        │
├─────────────────────────────────────┤
│  Command Execution Layer             │
└──────────────┬──────────────────────┘
               │
     ┌─────────┼─────────┐
     ▼         ▼         ▼
  [Tool1]   [Tool2]   [Tool3]
\end{verbatim}
\end{figure}

\textbf{Gap}: No standard for how agents compose and invoke their own CLIs.

\section{Rust Type System Innovations}

\subsection{Trait Objects for Runtime Polymorphism}
\begin{lstlisting}[language=Rust]
pub trait CommandHandler: Send + Sync {
    fn execute(&self, args: &CommandArgs)
        -> AgentResult<serde_json::Value>;
    fn metadata(&self) -> CommandMetadata;
}

pub type DynamicHandler = Arc<dyn CommandHandler>;
\end{lstlisting}

Enables runtime handler dispatch without monomorphization overhead.

\subsection{Zero-Cost Abstractions}
Rust's type erasure through trait objects provides:
\begin{itemize}
    \item No virtual method table overhead beyond one pointer indirection
    \item Compile-time safety with runtime flexibility
    \item Memory layout identical to manual dispatch implementations
\end{itemize}

\section{Related Work in Agent CLI Systems}

\subsection{Multi-Agent Coordination}
Recent work on agent teams (e.g., AutoGPT, Crewai) requires capability discovery:
\begin{quote}
\textit{``Current approaches require centralized capability registries or compile-time schema definitions.''} \cite{agentcoord2024}
\end{quote}

\textbf{Gap}: No lightweight runtime CLI generation mechanism.

\subsection{Semantic Web Integration}
RDF/Turtle and SPARQL provide rich semantic descriptions:
\begin{lstlisting}
:CreateCommand rdf:type :Verb ;
  rdfs:domain :File ;
  rdfs:range :FileSystem .
\end{lstlisting}

Our work integrates RDF-aware command discovery for agent negotiation.

\section{Performance Baselines}

\subsection{Compile-Time CLI Generation}
\begin{itemize}
    \item Generation overhead: 50-200ms (build time, amortized)
    \item Runtime lookup: 50-500ns per command
    \item Memory per command: 100-500 bytes (metadata)
\end{itemize}

\subsection{Our Runtime Approach}
\begin{itemize}
    \item Generation overhead: 38-40µs (per 64 commands)
    \item Runtime lookup: 200-600ns per command
    \item Memory per command: 100-500 bytes (identical)
\end{itemize}

This enables dynamic CLI updates previously impossible.
