\documentclass{article}
\usepackage{hyperref}
\usepackage{amsmath}
\usepackage{graphicx}

\title{Sample Narrative Paper}
\author{Playground CLI}
\date{\today}

\begin{document}
\maketitle

\begin{abstract}
This Narrative paper demonstrates the clap-noun-verb framework for building semantic CLIs.
\end{abstract}

% Narrative Thesis Structure: Field, Voice, Pattern, Insight
% This structure supports interpretive and qualitative research traditions

\section{Field}
Description of the research field.

Description of the research field and context, establishing the landscape within which the narrative unfolds.

\section{Voice}
Researcher's voice and perspective.

The researcher's perspective and positionality, acknowledging subjectivity and reflexivity in the interpretive process.

\section{Pattern}
Patterns identified in the data.

Identification and articulation of patterns emerging from the data, weaving together themes and recurring elements.

\section{Insight}
Insights and interpretations.

Interpretation and synthesis of insights, offering meaning-making and implications for understanding the phenomenon.

\bibliographystyle{plain}
\bibliography{references}

\end{document}
